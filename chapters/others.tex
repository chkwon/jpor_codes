\chapter{Useful and Related Packages}

The Julia Language is still a very early stage in its development. It was first released in 2012, and as of March 7, 2016 the stable version is 0.4.3---it is not yet 1.0. Python have a big user group and there are endless packages available. Python was first released in 1991 and the stable versions are 3.5.1 and 2.7.11.

One day, Julia will catch up, I believe. The number of Julia packages is fast increasing. The current list of registered Julia packages is available at \url{http://pkg.julialang.org}.

In this chapter, I selected some of packages that are useful and related to the tasks in operations research. Please note that some of these packages are very stable and likely remain available for long time, but some of these are very early in developments and could disappear at some point.

All of these Julia packages are available as a repository in \url{http://www.github.com}. For readers of the print version of this book, the best way of accessing the web page of each package is just googling with the package name. Online readers can just click the link at the package name.


\section{Basic Math}

Many basic mathematical functions are already included in the standard library of Julia. For math-related functions available, see the following official documentation:

\begin{itemize}
\item \href{http://docs.julialang.org/en/stable/stdlib/math/}{Mathematics Standard Library}\footnote{\url{http://docs.julialang.org/en/stable/stdlib/math/}}: functions for mathematics, statistics, signal processing, and numerical integration
\item \href{http://docs.julialang.org/en/stable/manual/linear-algebra/}{Linear Algebra Manual}\footnote{\url{http://docs.julialang.org/en/stable/manual/linear-algebra/}}: functions for matrix factorizations and special matrices
\item \href{http://docs.julialang.org/en/stable/stdlib/linalg/}{Linear Algebra Standard Library}\footnote{\url{http://docs.julialang.org/en/stable/stdlib/linalg/}}: standard functions for linear algebra, and functions from BLAS and LAPACK libraries
\end{itemize}

Beyond the standard library, the following packages would be useful:
\begin{itemize}
\item \href{https://github.com/JuliaLang/ODE.jl}{\kode{ODE.jl}}: solving ordinary differential equations
\item \href{https://github.com/JuliaLang/IterativeSolvers.jl}{\kode{IterativeSolvers.jl}}: solving linear systems
\item \href{https://github.com/JuliaLang/Roots.jl}{\kode{Roots.jl}}: finding roots of continuous functions of a single variable
\item \href{https://github.com/johnmyleswhite/Calculus.jl}{\kode{Calculus.jl}}: numerical differentiation / integration, and symbolic differentiation / simplification
\end{itemize}



\section{Optimization}

The main group of developers for optimization related packages is JuliaOpt with the website available at \url{http://www.juliaopt.org}. All of the packages are related to tasks in operations research directly and indirectly. Please visit the website of JuliaOpt; it gives a nice overview of available packages and optimization solvers for modeling and solving optimization problems in Julia.

Some additional packages related to general optimization modeling and algorithm are:
\begin{itemize}
\item \href{https://github.com/odow/NEOS.jl}{\kode{NEO.jl}}: The NEOS Server is a free online service for optimization solvers. When you don't have a license for a particular solver or don't have a machine to run a solver, the NEOS Server can be handy. Check their website at \url{http://www.neos-server.org/}. This package provides an interface between Julia and NEOS.
\item \href{https://github.com/IainNZ/JuMPeR.jl}{\kode{JuMPeR.jl}}: an extension of JuMP for Robust Optimization. This package can handle both linear and ellipsoidal sets of uncertainty.
\item \href{https://github.com/robertfeldt/BlackBoxOptim.jl}{\kode{BlackBoxOptim.jl}}: a global optimizer that supports multi-objective optimization problems and does not require objective functions to be differentiable.
\item \href{https://github.com/anriseth/MultiJuMP.jl}{\kode{MultiJuMP.jl}}: an extension of JuMP for multi-objective optimization problems. This package generates Pareto fronts.
\end{itemize}

\section{Graph and Network}

Two comprehensive packages related to Graph Theory and Network Optimization are:
\begin{itemize}
\item \href{https://github.com/JuliaLang/Graphs.jl}{\kode{Graphs.jl}}: Julia's standard package for graph types and algorithms. This package supports breadth first search, depth first search, shortest path algorithms, minimum spanning tree algorithms, random graph generation, etc.
\item \href{https://github.com/JuliaGraphs/LightGraphs.jl}{\kode{LightGraphs.jl}}: a similar package to \kode{Graphs.jl}, but simpler and light weighted.
\end{itemize}

Packages related to network optimization problems:
\begin{itemize}
\item \href{https://github.com/Azzaare/NetworkFlows.jl}{\kode{NetworkFlows.jl}}: This package provides network flow algorithms for solving max-flow problems, min-cut problems, etc.
\item \href{https://github.com/FugroRoames/Munkres.jl}{\kode{Munkres.jl}}: This package solves the optimal assignment problems using the Hungarian algorithm.
\end{itemize}

I have also written a few packages that are related to graph theory and network optimization:
\begin{itemize}
\item \href{https://github.com/chkwon/PathDistribution.jl}{\kode{PathDistribution.jl}}: This Julia package implements the Monte Carlo path generation method to estimate the number of simple paths between a pair of nodes in a graph (introduced in Section \ref{sec:pathdistribution}). This package also estimate the path-length distribution. That is, we can estimate the number of paths whose length is no greater than a certain number.
\item \href{https://github.com/chkwon/RobustShortestPath.jl}{\kode{RobustShortestPath.jl}}: This package solves robust shortest path problems for both single uncertain cost coefficient cases and two multiplicative uncertain cost coefficient cases.
\item \href{https://github.com/chkwon/TrafficAssignment.jl}{\kode{TrafficAssignment.jl}}: This package solves the traffic assignment problems and finds network user equilibrium traffic patterns. Available algorithms are Frank-Wolfe method, Conjugate Frank-Wolfe method, and Bi-conjugate Frank-Wolfe method.
\end{itemize}
You contributions to developing more functionalities and implementing more algorithms are welcome!



\section{Heuristics}

Meta-heurstic and heuristic algorithms:
\begin{itemize}
\item \href{https://github.com/wildart/Evolutionary.jl}{\kode{Evolutionary.jl}}: Evolutionary Strategies and Genetic Algorithms.
\item \href{https://github.com/WestleyArgentum/GeneticAlgorithms.jl}{\kode{GeneticAlgorithms.jl}}: Genetic Algorithms
\item \href{https://github.com/evanfields/TravelingSalesmanHeuristics.jl}{\kode{TravelingSalesmanHeuristics.jl}}: a heuristic algorithm for solving the Traveling Salesman Problem (TSP).  
\end{itemize}


\section{Statistics and Machine Learning}

A group of packages for statistics and machine learning is available via the JuliaStat group at \url{http://juliastats.github.io}. The group deals with issues regarding irregular data and array types and many standard to advanced methods in statistics and machine learning. The available packages and functions are nowhere near the completeness of those of \texttt{R}, but becoming mature and increasing in numbers quickly.

Packcage related to basic statistics:
\begin{itemize}
\item \href{https://github.com/JuliaStats/StatsBase.jl}{\kode{StatsBase.jl}}: basic functions for statistics including descriptive statistics and moments, counting and ranking, etc.
\item \href{https://github.com/JuliaStats/Distributions.jl}{\kode{Distributions.jl}}: a collection of probability distributions and maximum likelihood estimation for parameters
\item \href{https://github.com/JuliaStats/MultivariateStats.jl}{\kode{MultivariateStats.jl}}: linear regression, multidimensional scaling, etc.
\item \href{https://github.com/JuliaStats/HypothesisTests.jl}{\kode{HypothesisTests.jl}}: popular hypothesis tests handling confidence interval, $p$ value, and various parametric and non-parametric tests
\item \href{https://github.com/JuliaStats/TimeSeries.jl}{\kode{TimeSeries.jl}}: tools and functions for time series analysis
\end{itemize}

Packages using Markov-Chain Monte-Carlo (MCMC):
\begin{itemize}
\item \href{https://github.com/JuliaStats/Lora.jl}{\kode{Lora.jl}}: a generic engine for MCMC inference
\item \href{https://github.com/brian-j-smith/Mamba.jl}{\kode{Mamba.jl}}: MCMC for Bayesian analysis
\end{itemize}
There are several other different MCMC packages available. It seems that developers are trying to unify them under the \kode{MCMC.jl} package, someday.

Packages related to methods in Machine Learning:
\begin{itemize}
\item \href{https://github.com/JuliaStats/Clustering.jl}{\kode{Clustering.jl}}: various clustering algorithms, such as K-means, K-medoids, affinity propagation, and Density-based Spatial Clustering of Applications with Noise (DBSCAN).
\item \href{https://github.com/johnmyleswhite/kNN.jl}{\kode{kNN.jl}}: k-nearest neighbors (kNN) classification and regression
\end{itemize}
